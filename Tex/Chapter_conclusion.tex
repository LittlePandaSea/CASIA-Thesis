
\chapter{结束语与展望}
\label{Chapter_conclusion}

近年来,机器翻译研究取得了丰硕的成果,许多行之有效的翻译方法不断涌现,翻译性能也随之提高。从最初的基于词的模型,到目前方兴未艾的神经网络模型,机器翻译方法不断升级。在某些特定领域和场景下,机器翻译系统已经开始投入实际应用。但是由于模型和计算的复杂度,采用离散符号来表示翻译知识的统计机器翻译一般只考虑局部信息,并不考虑长距离的依赖关系,因而还存在长距离调序等难以回避的问题。新的神经网络机器翻译也面临着很多新的问题和挑战,需要人们进行更加深入的研究。因此,机器翻译自动译文质量还难以达到专业译员的期望,在人工翻译场景中的应用还面临方方面面的问题。基于翻译记忆的计算机辅助翻译软件仍然具有得天独厚的优势,但也遇到了瓶颈。

通过引入机器翻译来辅助专业译员以提高翻译效率是翻译行业的必然趋势。如果将机器翻译作为一种生产力工具,则译员对面向辅助翻译的机器翻译系统有更多的期待和更高的要求。较早被提出并且实现的译后编辑和交互式机器翻译也由于机器翻译自动译文质量还不够好而得不到广泛应用。因此本文以统计机器翻译为基础,致力于研究和改善人机交互式机器翻译,以提高人工翻译场景中的生产效率,同时提供良好的人机交互体验。

综上所述,本文的主要内容是围绕提高专业译员的生产效率,为人机交互式机器翻译提出更有针对性的解决方法。

\section{工作总结}

在详细、深入分析现有方法的优缺点基础之上,本论文取得了可观的研究成果:

1. 提出了一种融合统计机器翻译技术的中文输入方法

现实中,人们只用了机器翻译系统的自动译文。这种方式的缺点在于,如果自动译文的质量不能满足要求,则高质量的中间结果也一同被舍弃,从而使机器翻译难以充分发挥其价值。在大多数情况下,目前的机器翻译自动译文质量尚未达到人工翻译场景的用户期望,而已有的译后编辑和交互式机器翻译的人机交互体验过度依赖于自动译文质量。

为了充分发挥机器翻译在人工翻译场景中的作用,同时改善机器翻译人机交互体验,在该章中,我们提出的中文输入方法充分融合统计翻译中的翻译规则、翻译假设列表和翻译结果候选列表等相关信息,只需较少的按键次数就可以生成准确的译文结果。此外,为了指导统计机器翻译系统生成更适合该输入方法的翻译结果,我们提出了面向输入方法的译文自动评价指标。实验结果表明,该输入方法能大幅减少翻译人员的译文修改强度,显著提高翻译效率和译文质量。 同时,提出的自动评价指标能使该输入方法利用更合适的统计翻译结果,进一步提升人工翻译效率,显著改善人机交互体验。

2. 提出了一种一种基于术语识别边界信息的术语识别和翻译方法

术语翻译对于专业领域的机器翻译至关重要,而现有机器翻译系统往往没有专门考虑术语的翻译问题。因此,改善术语翻译对于专业领域中的人机交互式机器翻译至关重要。

为了改善专业领域中术语的翻译质量,我们提出了一种基于术语识别边界信息的术语识别和翻译方法。由于当前术语识别的性能还比较低,该方法借助术语识别边界信息建立术语解码方法,充分利用从平行句对和互联网单语语料中挖掘得到的术语翻译知识,包括三个部分:从平行句对中挖掘术语翻译知识的融合双语术语识别的联合词对齐模型,从单语语料中挖掘术语翻译知识的基于双语括号句子的术语翻译挖掘方法,以及基于术语识别边界信息的统计翻译术语解码方法。实验结果表明,我们提出的术语翻译方法能显著提升计算机领域专业术语的翻译准确率,从而有效地改善了统计翻译译文质量。

3. 提出了一种基于随机森林的统计翻译在线学习方法

重复纠正相同错误的乏味感让使用机器翻译的专业译员感到沮丧。这也是为什么现有的译后编辑和交互式机器翻译得不到专业译员的广泛支持的重要原因。人们期待机器翻译系统能在人机交互过程中实时学习来改善后续的自动译文。但是现有的基于用户反馈的机器翻译增量或者在线学习方法难以满足实际人工翻译过程的要求。

为使机器翻译系统能够在人机交互过程中有效利用译员已完成的双语句对,实时获取翻译知识并改善自动译文的质量,我们提出了一种基于随机森林的统计翻译在线学习方法。该方法通过在人机交互过程中实时从输入源文和用户反馈构成的平行句对中抽取翻译知识,不断更新基于随机森林的统计翻译模型,从而改善译文的质量。由于低频词和未登录词直接影响词对齐和翻译知识抽取的性能,因此,我们还提出了一种基于锚点的隐马尔可夫增量式词对齐方法。该词对齐方法有效利用互信息和词典等先验知识生成对齐锚点,然后联合执行基于锚点的双语短语划分和隐马尔可夫词对齐算法。模拟实验结果表明,随着用户反馈的积累,统计翻译在线学习方法显著提升了后续相关句子的自动译文质量, 且在线学习方法的译文质量可比于同等规模语料的离线学习基线系统的译文质量。人机交互体验得到显著改善。

最后,基于以上提出的方法,我们设计和实现了人机交互式英汉机器翻译系统,并总结了开发过程中遇到的关键问题和应对策略。

综上所述,本论文的研究工作以机器翻译在人工翻译场景中的应用为核心,有效地完善了人机交互式机器翻译方法,显著提升了人工翻译效率。本文的研究表明,通过为人机交互式机器翻译提出更有针对性的优化和改进措施,不仅可以有效改善机器翻译质量,还可以提升人工翻译效率,同时提供良好的人机交互体验。实际上,我们认为机器翻译与计算机辅助翻译不应该是各自平行独立发展的。虽然前者侧重于全自动翻译,后者侧重于通过人工翻译交付高质量译文。根据本文的初步研究工作表明,通过整合机器翻译与人工翻译的信息必然能更好地完成翻译任务。

直觉上也是如此,随着国际社会全球化进程的不断加快和国际交流的日益频繁,人们对不同语言之间的翻译需求越来越迫切。单独的机器翻译难以提供高质量的自动译文,而人工翻译的生产效率又无法满足快速膨胀的翻译需求。我们不认为存在机器翻译和人工翻译能否互相取代的问题,更重要的是如何通过二者的有机融合以更好地解决越来越重要的多语言交流需求。因此,本论文的研究工作将是后续研究的基础,如何进一步融合机器翻译与人工翻译知识并应用于解决实际的翻译需求将是未来机器翻译研究的重要方向。

\section{工作展望}

专门面向人工翻译场景的人机交互式机器翻译,是一个非常具有实用价值 和远景的研究工作。尽管本文已取得了很好的成果,但仍然还有很多工作值得我们进一步探索。我们下一步工作将主要关注以下几个方面:

1.中文输入方法与交互式机器翻译的进一步融合

虽然当前已经将统计机器翻译知识应用于中文输入方法,提升了人工翻译的生产效率,但在用户翻译过程中,统计机器翻译系统并不能根据用户已翻译部分针对剩余部分的译文进行进一步优化。但是很明显,我们应该能通过输入方法与交互式机器翻译方法的结合,使生成的译文提示更符合译员的期望,最后达到进一步提高翻译效率的目的。因此,如何融合中文输入方法与交互式机器翻译将是值得研究的一项工作。

2. 中文输入方法与神经网络机器翻译的融合问题

在目前的中文输入方法的对数线性模型中,我们仅能利用统计机器翻译系统的知识,还不能与神经网络机器翻译系统进行结合。虽然与专业译员的期望相比,神经网络机器翻译方法还存在着一些难以回避的问题。但是神经网络机器翻译已经达到甚至在多个方面超过了传统统计机器翻译的性能。因此,我们下一步将考虑在中文输入方法中引入神经网络机器翻译系统的知识,以提高神经网络机器翻译系统在人工翻译场景中的可用性。

3. 进一步探索术语翻译问题

由于我们的融合术语识别边界信息的术语翻译方法仅考虑了从平行句对中和网络上的双语括号句子中抽取术语翻译知识,术语翻译质量仍然有比较大的改进空间。融合术语识别边界信息的术语翻译方法的算法复杂度还比较高,因此,目前的术语翻译方法仍然有很多地方需要进一步优化和细化。

4. 充分利用专业译员翻译过程中的信息

目前已存在的机器翻译方法基本都没有考虑如何充分利用专业译员翻译过程中的信息,包括本文提出的基于在线随机森林的统计翻译模型在线学习方法也只用了最终完成的人工译文。并没有利用深层次的人工翻译过程信息,下一步我们将尝试引入这方面的特征。

